%import template
\input{./template.tex}

% DocInfo
\newcommand{\SUBJECT}{}
\newcommand{\TITLE}{Cheat Sheet Objektorientierte Programmierung}

\begin{document}

%import front page
% \input{./front.tex}

%do multicols
\begin{multicols*}{5}
    \setlength{\columnseprule}{0.4pt}
		\footnotesize
		
%import tableofcontents
% \input{./tableofcontents.tex}

\section{Allgemeines}
	\subsection{Primitive Datentypen}
		\begin{tabular}{p{.7cm} | p{2.1cm} | l}
			boolean&Boolescher Wert&true, false\\
			char&Textzeichen&'a','B','0','ë',...\\
			byte&Ganzzahl 8 Bit&-128 - 127\\
			short&Ganzzahl 16 Bit&-32'768 - 32'767\\
			int&Ganzzahl 32 Bit&-$2^{31}$ - $2^{31}$-1\\
			long&Ganzzahl 64 Bit&-$2^{63}$ - $2^{63}$-1, 1L\\
			float&Kommazahl 32 Bit&0.1f, 2e4f\\
			double&Kommazahl 64 Bit&0.1, 2e4\\
		\end{tabular}
	\subsection{Wrapper Klassen}
	\begin{lstlisting}
	// Einfach erster Buchstaben gross, ausser:
	// int -> Integer und char -> Character
	// Referenz-Typen -> for Generics
	// Can be null
	Integer boxed = Integer.valueof(5);
	int unboxed = boxed.intValue();
	\end{lstlisting}
		\subsubsection{Specials}
		\begin{lstlisting}
a[double - int] // cannot convert double to int
0/0 -> ArithmeticException div by zero
3.0/0 -> Double.POSITIVE_INFINITY
f + "" != f -> f ist jeder beliebige String
g+1 == g+2 -> g=Double.POSITIVE_INFINITY
h != h -> h = Float.NaN/Double.NaN
1 % 0 -> ArithmeticException div by zero
0 % 1 -> 0
int<long<float -> Compiler-Fehler
0.0 / 0.f -> NaN (Double.NaN)
x + 1 == x -> Double.POSITIVE_INFINITY
d++ == --d -> true
Math.abs(x - y) < delta // floats vergleichen
		\end{lstlisting}

	\subsection{Call by value/reference}
		\begin{itemize}
			\item Call by Value: Kopie von Wert/Referenz
			\item Call by Reference: Werte können verändert werden
			\item Java kann nur Call by Value
		\end{itemize}
	
	\subsection{Sichtbarkeit}
		\begin{itemize}
			\item Sichtbarkeit darf erweitert werden
			\item \tiny \textcolor{b}{\texttt{public}}
				\footnotesize : Alle Klassen
			\item \tiny \textcolor{b}{\texttt{protected}}
				\footnotesize : Packages und alle Sub-Klassen
			\item \tiny \textcolor{b}{\texttt{private}}
				\footnotesize : Nur eigene und äussere \textbf{Klasse} (nicht Objekt)
			\item \tiny kein Keyword
				\footnotesize : Alle Klassen im selben Package
		\end{itemize}
	
	
	\subsection{final}
		\begin{itemize}
			\item \textbf{Attribute}: nicht veränderbar
			\item \textbf{Methoden}: nicht überschreibbar
			\item \textbf{Klassen}: nicht erbbar
		\end{itemize}
	
	\subsection{equals}
		\begin{itemize}
			\item \tiny \textcolor{b}{\texttt{equals()}} 
				\footnotesize ist standardmässig nur Referenzvergleich
			\item Immer auch hashCode() überschreiben.
		\end{itemize}
\begin{lstlisting}
	a.equals(b) // 2 Strings
	Arrays.equals(a,b) // 2 Arrays a,b
	a.equals(b) // 2 ArrayLists
\end{lstlisting}
	
	\subsection{hashCode}
		\begin{itemize}
			\item Zwei gleiche Objekte = gleicher HashCode
			\item Umkehrung gilt nicht unbedingt
			\item Ungleiche Objekte können gleichen HashCode haben
		\end{itemize}
	\subsection{Overloading, Overriding \& Hiding}
	\begin{enumerate}
		\item Statischer Typ (links) für Overloading
		\item (null) $\rightarrow$ spezifischster wählen
		\item Dynamischer Typ (rechts) für Overriding
	\end{enumerate}
	\textbf{Overloading:} Gleiche Methode wird mit anderem Typ überschrieben. Die \textbf{spezifischere} Funktion wird gewählt. f spezifischer als g, bedeutet alle möglichen Anufrufe von f passen auch für g, aber nicht umgekehrt. -> Entscheidung bei Compilezeit \\
	\textbf{Overriding:} Gleiche Methode wird 1:1 überschrieben. -> Entscheidung in Laufzeit (nennt man Dynamic Call/Dispatch oder Virtual Call) (@Override) \\
	\textbf{Hiding:} Zwei Variablen mit selbem Namen in Klasse und Subklasse. Dann muss mit super.variable. super geht nur eine Stufe, sonst muss man mit z.b. ((Vehicle)this).variable zugreifen.
	
	\subsection{Exceptions}
	\begin{lstlisting}
String method(String s) throws Exception {
if (s == null) {
	throw new Exception("String is null"); 
} 
		\end{lstlisting}
		
		\subsubsection{Exceptions behandeln}
		\begin{itemize}
			\item Die Reihenfolge ist wichtig, die spezifizität ist egal. Der erste passende Catch block wird genutzt.
			\item Auch wenn ein Error im try/catch nicht gecatched wird, wird das finally ausgeführt.
		\end{itemize}
		\begin{lstlisting}
try {
	 /* regular code */ 
}  catch (Exception e) {
	/* error handling */
} finally {
	/* e.g. cleanup */ 
}
		\end{lstlisting}
		\subsubsection{Eigene Exceptions}
		\begin{lstlisting}
class StringClipException extends Exception {
	public StringClipException() {}
	public StringClipException(String message) {
		super(message); }}
//------------------in Methode:------------------
String clip(String s) throws StringClipException {
	if (s == null) {
	throw new StringClipException("no string"); } }
		\end{lstlisting}
		\subsubsection{Checked und Unchecked Exceptions}
		\textcolor{b}{Checked Exceptions:}
		\begin{itemize}
			\item Exception muss behandelt werden (catch)
			\item \textbf{throws}-Deklaration im Methodenkopf
			\item Wird vom Compiler geprüft
			\item \tiny \texttt{ClassNotFoundExc., IllegalAccessExc., EigeneExc.}
		\end{itemize}
		\textcolor{b}{Unchecked Exceptions}
		\begin{itemize}
			\item Keine throws-Deklaration/Behandlung nötig
			\item Wird nicht vom Compiler geprüft
			\item Also alle Subtypes von \textbf{RuntimeException} (NullPointerException, IndexOutOfBoundsException, ...) und  \textbf{Error} (OutOfMemory, StackOverflow, ...)
		\end{itemize}
		\subsubsection{Exceptions bei vererbten Methoden (Overrides)}
		\begin{lstlisting}
// Basisklasse/Interface
void method() throws ExceptionType
//Subklasse: 4 Possible Overrides
void method() throws ExceptionType //(same)
void method() // Weglassen
void method() thorws SubExceptionType //Subtype
void method() throws RunTimeException //unchecked
		\end{lstlisting}
	
\section{Vererbung}
	\subsubsection{Abstract Class}
	\begin{lstlisting}
public abstract class Vehicle {
	//Vererbbare Methoden....
	public abstract void drive();
	\end{lstlisting}
	\subsubsection{Erbende Klassen von Klassen}
	\begin{lstlisting}
public class Car extends Vehicle {
	@Override
	public void drive() {
	System.out.println("Car");
	//Ausfuehren der Vatermethode wenn nicht abstract
	super.drive(); 
	\end{lstlisting}
	\subsubsection{Erben von Interfaces}
	\begin{lstlisting}
public class RegularCar implements Vehicle {...}
	\end{lstlisting}




\section{Interfaces}
	\subsection{Allgemeines}
	\begin{itemize}
		\item Facette einer Klasse
		\item Trennung Spezifikation und Implementation
		\item Lollipop oder Hierarchische Mehrfachvererbung
		\item Nur Klassen sind instanzierbar
		\item Schnittstellen sind aber als statische Typen verwendbar (linke Seite)
		\item Variablen in Interfaces sind Konstanten!
		\item Keine Instanzvariablen im Interface
	\end{itemize}
	\subsection{Vererbung}
		\subsubsection{Interface $\rightarrow$ Interface}
			\begin{lstlisting}
public interface Vehicle{...}
public interface LandMobile extends Vehicle{...}
			\end{lstlisting}
		\subsubsection{Interface $\rightarrow$ Klasse}
			\begin{lstlisting}
public interface Vehicle{...}
public class Car implements Vehicle{...}
			\end{lstlisting}
	
	\subsection{Default Methods}
	\begin{itemize}
		\item Falls Kollisionen bei Mehrfachvererbung 
		\item Ähnlich zu Mehrfachvererbung von abstrakten Klassen
		\item Keine Instanzvariablen in Schnittstelle!
		\item Methoden können in Klasse überschrieben werden
		\item Falls nicht überschrieben: Defaultmethode
	\end{itemize}
		\begin{lstlisting}
interface Vehicle {
	public default void printModel() {...method...}
		\end{lstlisting}


	\subsection{Enum}
\begin{lstlisting}
	public enum Weekday {
		MONDAY(true), TUESDAY(true), WEDNESDAY(true), THURSDAY(true), FRIDAY(true), SATURDAY(true), SUNDAY(true)
		
		public boolean workDay;		
		Weekday(boolean workDay){
			this.workDay = workDay;
		}
		public boolean isWorkDay() {
			return workDay;
		}
	}
\end{lstlisting}


	\section{Collections}
	\subsection{Array}
		\begin{lstlisting}
	// possible array inits:
	var array = new int[] { 1, 2, 3};
	int array[] = { 1, 2, 3 };
	var array = new int[3]; // empty
		\end{lstlisting}
	
	\subsection{List}
		\subsubsection{ArrayList}
			\begin{itemize}
				\item Duplikate möglich
				\item null ist auch einfügbar
				\item Add/Remove während Iterierens $\rightarrow$ Fehler
				\item get(), set(), add(): sehr schnell
				\item remove(), contains(): langsam
			\end{itemize}
			\begin{lstlisting}
List<String> stringList = new ArrayList<>();
List.of(array); // Create list out of array
	stringList.add("OO"); //Add Element
	stringList.add(0, "Bsys1"); //Add @ pos 0
	String x = stringList.get(1); //get @ pos 1
	stringList.set(0, "Bsys2"); //replace at pos 0
	boolean b = stringList.contains("CN1")
	stringList.remove("ICTh");
	stringList.remove(1); //remove @ pos 1
	people.removeAll(group, x -> x.getAge() > 65);
			\end{lstlisting}

		\subsubsection{LinkedList}
		\begin{itemize}
			\item Verkettete Liste der Elemente
			\item Duplikate möglich
			\item Dynamisch hinzufügbar und entfernbar
			\item Intern Doppelt verkettet
			\item add(), remove(anfang/ende) sehr schnell
			\item get(), set(), contains(), remove() langsam
		\end{itemize}
		\begin{lstlisting}
List<String> firstList = new LinkedList<>();
	//Gleiche Funktionen wie ArrayList
		\end{lstlisting}
	
		\subsubsection{Queue / Deques}
		\begin{itemize}
			\item Verkettete Liste der Elemente
			\item Dynamisch hinzufügbar und entfernbar
			\item LIFO oder FIFO möglich
			\item Intern Doppelt verkettet
		\end{itemize}
		\begin{lstlisting}
ArrayDeque<Type> animal = new ArrayDeque<>();
queue.addLast(elem);
queue.addFirst(elem);
queue.removeFirst(elem);
queue.removeLast(elem);
queue.clear(); //remove all
queue.peek() //Get first el, counter to peekLast()
		\end{lstlisting}

	\subsubsection{Stack}
	\begin{lstlisting}
//Better to use Deque instead!
var stack = new Stack<E>();
var pushedItem = stack.push(el);
var poppedEl = stack.pop(); //get and remove top el
var el = stack.peek() //get top el (no remove)
var isEmpty = stack.empty();
stack.isEmpty(), stack.isFull() //2 booleans
int pos = stack.search("Test") // Pos of elem
	\end{lstlisting}
	
	\columnbreak

	\subsection{Set}
		\begin{itemize}
			\item Für loop: enhanced for verwenden
		\end{itemize}
		\subsubsection{HashSet}
		\begin{itemize}
			\item Keine Duplikate
			\item Unsortiert 
			\item \textbf{Oft} sehr effizient
			\item In Hashtabelle gespeichert
			\item Elemente geben hashCode() konsistent zu equals()
		\end{itemize}
		\begin{lstlisting}
Set<String> otherSet = new HashSet<>();
Set<String> s = new HashSet<>(list); //from list
	set.add(elem); //Add one
	set.remove(elem); //Remove one
	set.size(); //Get Length
	set.addAll(list); //Add Collection of elem
	set.isEmpty(); // Empty?
	set.contains(elem) // Contains
	String[] a = (String[]) set.toArray(); //Convert
		\end{lstlisting}
		\subsubsection{TreeSet}
		\begin{itemize}
			\item Keine Duplikate
			\item Sortiert
			\item \textbf{immer} effizient
			\item In Binärbäumen gespeichert
			\item Elemente implementieren Comparable und equals()
		\end{itemize}
		\begin{lstlisting}
Set<String> firstSet = new TreeSet<>();
	//Gleiche Funktionen wie HashSet
		\end{lstlisting}
	
	\subsection{Map}
		\subsubsection{TreeMap}
		\begin{itemize}
			\item Mengen von Schlüssel-Wert-Paaren
			\item Nach Schlüssel sortiert
			\item immer effizient
		\end{itemize}
		\begin{lstlisting}
Map<Integer, String> bachelors = new TreeMap<>();
	map.put(20000, "Hello");
	map.containsKey(key);
	map.containsValue(value);
	String x = map.get(20000);
	for (int number : map.keySet()) { 
		System.out.println(number);
	}
	for (String value : map.values()) { 
		System.out.println(value);
	}
	for (var entry: map.entrySet()){
		//entry.getKey() and entry.getValue()
	}
		\end{lstlisting}
		\subsubsection{HashMap}
		\begin{itemize}
			\item Mengen von Schlüssel-Wert-Paaren
			\item Unsortiert
			\item oft sehr effizient
		\end{itemize}
		\begin{lstlisting}
Map<Integer, String> map = new HashMap<>();
// Gleiche Funktionen wie TreeMap
		\end{lstlisting}
	\subsection{Iterator}
	\begin{lstlisting}
	var it = stringList.iterator();
	while (it.hasNext()){
		String elem = it.next();
		if(elem.equals("test123")){
			it.remove(); // geht nicht bei enhanced for
		}
	}
	\end{lstlisting}

\section{Records}
	\begin{itemize}
		\item Wie eine Klasse mit automatisch erstelltem Konsturktur, equals, hashCode, toString und getter (keine Setter, da nicht veränderbar)
		\item Kann noch mehr Methoden enthalten
		\item Konstruktur vom Compiler kann werden
		\item Statische Vars erlaubt, Instanz-Vars nicht
	\end{itemize}
	\begin{lstlisting}
	public record Point(int x, int y) {
		private static final int INITIAL = 0;
		
		public Point() {
			this(INITIAL, INITIAL);
		}
	}
	\end{lstlisting}

\section{Annotations}
	\begin{itemize}
	\item Starten mit "@", Infos fü den Compiler
	\item Aendern nicht was der Code macht
	\item Der Compiler verarbeitet diese durch Annotation Processors. So kann auch neuer Sourcecode oder so erzeugt werden (z.B. mit Lombok).
\end{itemize}

\section{Stream API}
	\subsection{Zwischenoperationen}
		\begin{itemize}
			\item \tiny \textcolor{b}{\texttt{filter(Predicate)}}
				\footnotesize Rauspicken gem. Predicate
			\item \tiny \textcolor{b}{\texttt{map(Function)}}
				\footnotesize Projizieren gemäss Function
			\item \tiny \textcolor{b}{\texttt{mapToInt(Function)}}
				\footnotesize Proji. auf primitiver Typ
			\item \tiny \textcolor{b}{\texttt{sorted()}}
				\footnotesize Sortieren
			\item \tiny \textcolor{b}{\texttt{distinct}}
				\footnotesize Duplikate werden gelöscht
			\item \tiny \textcolor{b}{\texttt{limit(long n)}}
				\footnotesize Erste n Elemente liefern
			\item \tiny \textcolor{b}{\texttt{skip(long n)}}
				\footnotesize Erste n Elemente ignorieren
		\end{itemize}
	\subsection{Terminaloperationen}
		\begin{itemize}
			\item \tiny \textcolor{b}{\texttt{forEach(Consumer)}}
				\footnotesize Pro Element anwenden
			\item \tiny \textcolor{b}{\texttt{forEachOrderd(consumer)}}
				\footnotesize Erhält die Reihenfolge der Elemente
			\item \tiny \textcolor{b}{\texttt{count()}}
				\footnotesize Anzahl Elemente als long
			\item \tiny \textcolor{b}{\texttt{min(), max()}}
				\footnotesize Mit Comparator Argument ausser schon korrekt gemapped. Als Optional.
			\item \tiny \textcolor{b}{\texttt{average()}}
				\footnotesize Nur bei int, long, double. Optional.
			\item \tiny \textcolor{b}{\texttt{sum()}}
				\footnotesize Nur bei int, long, double.
			\item \tiny \textcolor{b}{\texttt{findAny()}}
				\footnotesize Gibt irgendein Element als Optional
			\item \tiny \textcolor{b}{\texttt{findFirst()}}
				\footnotesize Gibt erstes Element als Optional
			\item \tiny \textcolor{b}{\texttt{allMatch(),anyMatch(),noneMatch()}}
			\footnotesize Prüfen ob auf alle/irgendein/keine Prädikate etwas zutrifft.
		\end{itemize}
	\subsection{Optional}
		\begin{itemize}
			\item \tiny \textcolor{b}{\texttt{isPresent()}}
				\footnotesize true wenn Element vorhanden
			\item \tiny \textcolor{b}{\texttt{isEmpty()}}
				\footnotesize true wenn \textbf{kein} Element vorhanden
			\item \tiny \textcolor{b}{\texttt{get()}}
				\footnotesize Gibt Element. \textbf{Exception} wenn empty.
		\end{itemize}
	\subsection{Rückumwandlung}
		\textcolor{b}{Zu Array:}
		\begin{lstlisting}
Person[] arr = peopleStream.toArray(Person[]::new);
		\end{lstlisting}
		\textcolor{b}{Zu List}
		\begin{lstlisting}
List<Person> list = peopleStream.toList();
		\end{lstlisting}	
	
		\subsection{Lambda, Stream Beispiele}
	\begin{lstlisting}
Arrays.stream(people) // Array zu stream
people.stream() //Collections zu streams;
IntStream.range(1, 100) // Zahlen von bis 100
IntStream.iterate(2, i -> i + 1); // Zahlen ab 2
Stream.of(2,3,5,7) // Eigene Aufzaehlung
Stream.concat(stream1, stream2) // 1 gefolgt von 2
stream().forEach(System.out::println); //print all

stream().sort((p1, p2) -> { 
	return Integer.compare(p1.getAge(), p2.getAge());
	//Oder
	return p1.getName().compareTo(p2.getName());
});

stream().filter(p -> p instanceof House)
stream().map(p -> p.getName()) // to String
stream().mapToInt(i -> i).sum(); // Sum
	\end{lstlisting}
	\begin{lstlisting}
		
// Top 10 earner
people.stream()
.sorted(Comparator.comparing(
Person::getSalary).reversed())
.mapToInt(p -> p.getSalary()).limit(10)

//Multiple comparisons
.sorted(Comparator.coparing(...).thenComparing(...))

// get Max Age from People in Zurich
people.stream()
.filter(p -> p.getCity().equals("Zuerich"))
.mapToInt(p -> p.getAge()).max().getAsInt();

// get Average Age of Male
var x = people.stream()
.filter((p) -> p.getGender().equals(Gender.MALE))
.mapToInt(p -> p.getAge()).average();

// Get all female Names with 3 or less Characters
people.stream()
.filter((p) -> p.getGender().equals(Gender.FEMALE) 
&& p.getFirstName().length() <= 3)
.map(p -> p.getFirstName())

// Durchschnittsalter pro Ort (mittels Collector)
people.stream()
.collect(Collectors.groupingBy(Person::getCity, Collectors.averagingInt(Person::getAge)))
.forEach((city, age) -> System.out.println(city));

// Gruppieren
Map<Integer, List<Person>> peopleByAge = 
people.stream()
.collect(Collectors.groupingBy(Person::getAge));
		
	\end{lstlisting}

\section{Beispiele}

	\subsubsection{Switch - Case}
	\begin{lstlisting}
int x = 5; 
switch(x) {
	case 2: 
		System.out.println("2"); 
		break;
	case 5: 
		System.out.println("5"); 
		break;
	default: 
		System.out.println("No match!"); 
}
	\end{lstlisting}

	\subsubsection{for loop}
	\begin{lstlisting}
	for(int i = 0; i < list.size(); i++) {}
	for(var el : list) {}
	for(var el : set) {}
	for(var el : map.entrySet()) {
		var k = el.getKey();
		var v = el.getValue();
	}
	\end{lstlisting}

	\subsubsection{Main Methode}
	\begin{lstlisting}
	public static void main(String[] args) {...}	
	\end{lstlisting}

	\subsubsection{Interface}
	\begin{lstlisting}
	public interface GraphicItem {
		Vector getLocation();
		void move(Vector delta);
		
		// A default implementation:
		default void moveToCenter() {
			var pos = getLocation();
			move(new Vector(-pos.getX(), -pos.getY()));
		}
	}
	\end{lstlisting}

	\subsubsection{Substring / Split}
		\begin{lstlisting}
int number = Integer.parseInt(line.substring(0, 3));
String name = line.substring(4);
String a = "1234 10"; 
String[] c = a.split(" ");
		\end{lstlisting}


	\subsubsection{Podcast}
	\begin{lstlisting}
class Chapter {
	// Constructor..., Attributes...
	public int getLength() {
		return content.length;
	}
}
public class Podcast {
	// Constructor..., Attributes...
	public int getLength() {
		int sum = 0;
		for (var chapter : chapters) {
			sum += chapter.getLength();
		}
		return sum;
	}
}
public class PodcastMaker {
	static ArrayList<byte[]> splitIntoChapters(byte[] recording, int[] chapterMarks) {
		var result = new ArrayList<byte[]>();
		var previousChapterEnd = 0;
		
		for (int nextChapterEnd : chapterMarks) {
			var nextChapter = Arrays.copyOfRange(recording, previousChapterEnd, nextChapterEnd);
			result.add(nextChapter);
			previousChapterEnd = nextChapterEnd;
		}
		
		var lastChapter = Arrays.copyOfRange(recording, previousChapterEnd, recording.length);
		result.add(lastChapter);
		return result;
	}
	static Podcast createPodcast(ArrayList<byte[]> chapters) {
		var chapterObjects = new ArrayList<Chapter>();
		for (var chapter : chapters) {
			chapterObjects.add(new Chapter(chapter));
		}
		return new Podcast(chapterObjects);
	}
}
	\end{lstlisting}

	\subsubsection{toString mit Indent}
	\begin{lstlisting}
	// Vaterklasse:
	public abstract String toString(int indent);			
	
	@Override
	public String toString() {
		return toString(0);}
	
	protected String getIndent(int length) { 
		String text = ""; 
		for (int i = 0; i < length; i++) { 
			text += " ";
		}
		return text;
	}
	// Erbende Klasse:
	@Override
	public String toString(int indent) {
		String text = getIndent(indent) + getName() +"\n";
		for (Food food : ingredients) {
			text += food.toString(indent + 1);
		}
		return text; }
	\end{lstlisting}

	\subsubsection{Erastothenes}
		\begin{lstlisting}
// The numbers in the sieve start with 2,
// so we reduce the array length accordingly.
int[] sieve = new int[PRIMES_UP_TO - 1];
for(int i = 0; i < sieve.length; i++) {
	sieve[i] = i + 2;
}

// Now let's look at each number in the array:
for(int i = 0; i < sieve.length; i++) {
	int primeNumber = sieve[i];
	
	// Has the number already been crossed out?
	if(primeNumber == -1) {
		continue;
	}
	
	int numberToCross = primeNumber * 2;
	while(numberToCross <= PRIMES_UP_TO) {
		// Since the numbers and indices differ
		// by two, we have to calculate the index:
		int indexToCross = numberToCross - 2;
		// An int[] always needs to contain a valid
		// int, we can't really cross an item, but
		// instead set a special value:
		sieve[indexToCross] = -1;
		// And we continue with the next number:
		numberToCross += primeNumber;
	}
}
		\end{lstlisting}
	
	
	\subsubsection{Custom 'Linked List'}
	\begin{lstlisting}
public class Car {
	private Car next;
	// ...
}
public class Locomotive {
	private Car first;
	// ...
}
public class Train {
	private Locomotive locomotive;
	public Train() {
		locomotive = new Locomotive();
	}
	
	public void add(String name) {
		Car first = locomotive.getFirst();
		Car newCar = new Car(name);
		newCar.setNext(first);
		locomotive.setFirst(newCar);
	}
	
	public boolean insert(int position, String name) {
		if (position == 0) {
			add(name);
			return true;
		}
		Car leadingCar = getCar(position - 1);
		if (leadingCar == null) {
			return false;
		}
		Car followingCar = leadingCar.getNext();
		Car newCar = new Car(name);
		leadingCar.setNext(newCar);
		newCar.setNext(followingCar);
		return true;
	}
	
	public void reverse() {
		Car previousCar = null;
		Car currentCar = locomotive.getFirst();
		while (currentCar != null) {
			Car nextCar = currentCar.getNext();
			currentCar.setNext(previousCar);
			previousCar = currentCar;
			currentCar = nextCar;
		}
		locomotive.setFirst(previousCar);
	}
	
	public Car removeFirst() {
		Car firstCar = locomotive.getFirst();
		if (firstCar != null) {
			locomotive.setFirst(firstCar.getNext());
		}
		return firstCar;
	}
}
	\end{lstlisting}
	
	\subsection{compareTo}
	
	\subsubsection{Compare To}
		\begin{lstlisting}
//Dieses Interface gibts in Java schon
@FunctionalInterface
interface Comparable<T> {
	int compareTo(T other); }

class Person implements Comparable<Person> {
	private int age;
	@Override 
	public int compareTo(Person other) {
		return Integer.compare(age, other.age); }
	//oder
	@Override 
	public int compareTo(Person other) {
		int result = lastName.compareTo(other.lastName);
		if (result == 0) {
			result = firstName.compareTo(other.firstName); } 
		return result;}}
		//oder
		people.sort(this::compareByAge);//in Methode
		//oder
		(p1, p2) -> p1.compareTo(p2)
		\end{lstlisting}
	
	
	
	\subsubsection{Comparator}
		\begin{lstlisting}
class AgeComparator implements Comparator<Person> {
	@Override
	public int compare(Person first, Person second){
		return Integer.compare(first.getAge(), second.getAge() ); }
		\end{lstlisting}
	
	\subsubsection{Comparable}
		\begin{lstlisting}
class Person implements Comparable<Person> {
	//.....
	@Override
	public int compareTo(Person other){
		int result = lastName.compareTo(other.lastName);
		if (result == 0){ 
		 result = firstName.compareTo(other.firstName);}
		return result;}
		\end{lstlisting}

	\subsection{equals}
		
		\begin{lstlisting}
@Override
public boolean equals(Object obj) {
	if (obj == null) {
		return false;
	}
	if (getClass() != obj.getClass()) {
		return false;
	}
	if (!super.equals(obj)) {
		return false;
	}
	Person other = (Person) obj;
	return Objects.equals(getFirstName(), other.getFirstName()) 
		&& Objects.equals(getLastName(), other.getLastName());
}
		\end{lstlisting}
	
	\subsection{hashCode}
		\begin{lstlisting}
@Override
public int hashCode() {
	return Objects.hash(firstName, lastName);
}
		\end{lstlisting}	
	\subsection{Diverses Rekursives}
	\subsubsection{Palindrom}
	\begin{lstlisting}
printPalindroms("", "", 0, 5); //in Main

void printPalindroms(String prefix, String suffix, int pos, int length) {
	if (pos == (length + 1) / 2) {
		System.out.println(prefix + suffix);
	} else {
		boolean middle = position == length / 2;
		for (char c = 'A'; c <= 'Z'; c++) {
			printPalindroms(prefix + c, middle ? 
			suffix : c + suffix, position + 1, length);
		}}}
	\end{lstlisting}
	\subsubsection{Max-Clique (with Subsets of Set)}
\begin{lstlisting}
Set<Person> maxClique() {
	Set<Person> result = new HashSet<>();
	for (Set<Person> group : allSubsets( new ArrayList<>(allPeople))) {
		if (isClique(group) && result.size() < group.size()) {
			result = group;} } return result; }

boolean isClique(Set<Person> group) { 
	for (Person first : group) {
		for (Person second : group) {
			if (first != second && !first.getFriends().contains(second)) {
				return false; 
			} 
		} 
	} 
	return true; 
}

Set<Set<Person>> allSubsets(List<Person> people) { 
	Set<Set<Person>> result = new HashSet<>(); 
	result.add(new HashSet<>());
	for (Person current : people) {
		List<Person>remaining = new ArrayList<>(people); 
		remaining.remove(current);
		for (Set<Person>subset : allSubsets(remaining)){
			result.add(subset);
			Set<Person>augmented =new HashSet<>(subset); 
			augmented.add(current); 
			result.add(augmented);
		} 
	} return result; 
}
	
\end{lstlisting}
\subsubsection{Fakultät Rekursiv}
\begin{lstlisting}
static long fact(long i){
	if(i==0){return 1;}
	else{return i * fact(i -1);}
}
\end{lstlisting}
	\subsubsection{Alle möglichen Buchstabenkombinationen}
	\begin{lstlisting}
Set<String> generateWords(Set<Character> input) {
	Set<String> result = new HashSet<>();
	if (input.size() == 0) {
		result.add("");
	}
	for (Character letter : input) {
		Set<Character> remainder = new HashSet<>(input);
		remainder.remove(letter);
		for (String word : generateWords(remainder)) {
			result.add(word + letter);
		}
	}
	return result;
}
	\end{lstlisting}
\subsection{Noch mehr Zeugs :-)}
	\subsubsection{Cluster von Personen}
\begin{lstlisting}
boolean isLinked(Person from, Person to){
	return isLinked(from, to, new HashSet<>()); }

boolean isLinked (Person from, Person to, 
Set<Person> visited) {
	if (from == true) { return true;}
	if (visited.contains(from)){
		return false; //dont check twice
	}
	visited.add(from);
	for ( Person other: from.knownPeople()){
		if (isLinked(other, to, visited)){
			return true	
		} 
	}
	return false; 
}
\end{lstlisting}
\subsubsection{Optimum an Elementen finden}	
\begin{lstlisting}
Set<Item> solveRucksack(List<Item> allItems, 
int maxWeight) { 
	Set<Set<Item>>allCombinations =powerSet(allItems); 
	Set<Item> optimum = new HashSet<>();
	for (Set<Item> combination: allCombinations) {
		if (totalWeight(combination) <= maxWeight && 
		totalValue(combination)>totalValue(optimum)){
			optimum = combination; } } 
	return optimum; }

Set<Item> solve(List<Item> allItems, int maxWeight){ 
	Set<Set<Item>>allCombinations =powerSet(allItems); 
	Set<Item> optimum = new HashSet<>(); //result
	for (Set<Item> combination: allCombinations) {
		if (totalWeight(combination) <= maxWeight && 
		totalValue(combination) >totalValue(optimum)){
			optimum = combination; }} return optimum;}

int totalValue(Set<Item> combination) {
	return combination.stream().mapToInt(item -> item.getValue()).sum();}

int totalWeight(Set<Item> combination) {
	return combination.stream().mapToInt(item -> item.getWeight()).sum();}
\end{lstlisting}
\subsubsection{Shortest Roundtrip}	
\begin{lstlisting}
List<City> shortestRoundtrip(List<City> visited) {
	if (visited.size() == allCities.size()) {
		return visited;
	}
	int min = Integer.MAX_VALUE;
	List<City> best = null;
	for (City next : allCities) {
		if (!visited.contains(next)) {
			List<City> expanded = new ArrayList<>(visited);
			expanded.add(next);
			List<City> result = shortestRoundtrip(expanded);
			int distance = totalDistance(result);
			if (distance < min) {
				min = distance;
				best = result;
			}
		}
	}
	return best;
}
int totalDistance(List<City> roundtrip) {
	int sum = 0;
	for (int index = 0; index < roundtrip.size(); index++) {
		City current = roundtrip.get(index);
		City next = roundtrip.get((index + 1) % roundtrip.size());
		sum += current.getConnections().get(next);
	}
	return sum;
}
//Verwendung:
List<City> shortest = shortestRoundtrip(Arrays.asList(startCity));
System.out.println(shortest);
System.out.println(totalDistance(shortest));
	\end{lstlisting}
	
\subsection{Viel Glück}	
\begin{lstlisting}
         r==
     _  //
    |_)//(''''':
      //  \_____:_____.-----.P
  	 //   | ===  |   /        \
 .:'//.   \ \=|   \ /  .:'':.
:' // ':   \ \ ''..'--:'-.. ':
'. '' .'    \:.....:--'.-'' .'
 ':..:'                ':..:'
 
\end{lstlisting}
Java, ach Java, so mächtig und groß,\\
doch manchmal fühlt es sich an wie ein Floh. \\
Mit jeder Zeile, die ich tippe und dreh',\\
wächst der Wunsch, dass es einfacher geh'.\\

Die Syntax streng, die Fehler so klar,\\
manchmal scheint Java fast unerklärbar.\\
Exceptions werfen, fangen, behandeln,\\
will ich doch nur programmieren, nicht rätseln und rändeln.\\

Objektorientierung, ein hehres Ziel,\\
doch manchmal erscheint es mir zuviel.\\
Klassen, Vererbung, zu viel des Guten,\\
kann nicht einfach alles in Funktionen ruhen?\\
\end{multicols*}

% \input{./appendix.tex}

\end{document}